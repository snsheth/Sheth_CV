\documentclass[11pt,english]{article}\usepackage[]{graphicx}\usepackage[]{xcolor}
% maxwidth is the original width if it is less than linewidth
% otherwise use linewidth (to make sure the graphics do not exceed the margin)
\makeatletter
\def\maxwidth{ %
  \ifdim\Gin@nat@width>\linewidth
    \linewidth
  \else
    \Gin@nat@width
  \fi
}
\makeatother

\definecolor{fgcolor}{rgb}{0.345, 0.345, 0.345}
\newcommand{\hlnum}[1]{\textcolor[rgb]{0.686,0.059,0.569}{#1}}%
\newcommand{\hlstr}[1]{\textcolor[rgb]{0.192,0.494,0.8}{#1}}%
\newcommand{\hlcom}[1]{\textcolor[rgb]{0.678,0.584,0.686}{\textit{#1}}}%
\newcommand{\hlopt}[1]{\textcolor[rgb]{0,0,0}{#1}}%
\newcommand{\hlstd}[1]{\textcolor[rgb]{0.345,0.345,0.345}{#1}}%
\newcommand{\hlkwa}[1]{\textcolor[rgb]{0.161,0.373,0.58}{\textbf{#1}}}%
\newcommand{\hlkwb}[1]{\textcolor[rgb]{0.69,0.353,0.396}{#1}}%
\newcommand{\hlkwc}[1]{\textcolor[rgb]{0.333,0.667,0.333}{#1}}%
\newcommand{\hlkwd}[1]{\textcolor[rgb]{0.737,0.353,0.396}{\textbf{#1}}}%
\let\hlipl\hlkwb

\usepackage{framed}
\makeatletter
\newenvironment{kframe}{%
 \def\at@end@of@kframe{}%
 \ifinner\ifhmode%
  \def\at@end@of@kframe{\end{minipage}}%
  \begin{minipage}{\columnwidth}%
 \fi\fi%
 \def\FrameCommand##1{\hskip\@totalleftmargin \hskip-\fboxsep
 \colorbox{shadecolor}{##1}\hskip-\fboxsep
     % There is no \\@totalrightmargin, so:
     \hskip-\linewidth \hskip-\@totalleftmargin \hskip\columnwidth}%
 \MakeFramed {\advance\hsize-\width
   \@totalleftmargin\z@ \linewidth\hsize
   \@setminipage}}%
 {\par\unskip\endMakeFramed%
 \at@end@of@kframe}
\makeatother

\definecolor{shadecolor}{rgb}{.97, .97, .97}
\definecolor{messagecolor}{rgb}{0, 0, 0}
\definecolor{warningcolor}{rgb}{1, 0, 1}
\definecolor{errorcolor}{rgb}{1, 0, 0}
\newenvironment{knitrout}{}{} % an empty environment to be redefined in TeX

\usepackage{alltt}
\usepackage[T1]{fontenc}
\usepackage{needspace}
\usepackage[latin9]{inputenc}
\usepackage[a4paper]{geometry}
\geometry{verbose,tmargin=1in,bmargin=1in,lmargin=1in,rmargin=1in}
%\setlength{\parskip}{\smallskipamount} % removing this line helped problem with extra page breaks between sections
\setlength{\parindent}{0pt}
\usepackage{array}

\makeatletter

\providecommand{\tabularnewline}{\\}

%%%%%%%%%%%%%%%%%%%%%%%%%%%%%% User specified LaTeX commands.
\makeatother
\usepackage{lineno,setspace}
\usepackage{fancyhdr}
\usepackage{lastpage}
\pagestyle{fancy}
\usepackage[small,compact]{titlesec}
\usepackage[small,it]{caption}
\addtolength{\belowcaptionskip}{-3mm}
\addtolength{\abovecaptionskip}{-3mm}
\usepackage{hanging}
\usepackage{babel}
\usepackage[hidelinks]{hyperref}
\usepackage{xcolor}
\usepackage{lipsum}
\usepackage{tabularx}
\newcommand\reverselabel[1]{%
  \def\theenumi{}%
  \renewcommand\makelabel{\makebox[\dimexpr\labelwidth-3pt\relax][r]{%
    \the\numexpr#1-\value{enumi}+1\relax}}}%

\makeatother

\fancyhead{}
\renewcommand{\headrulewidth}{0pt}
\fancyfoot{}
\lfoot{Sheth CV}
%\cfoot{CV}
\rfoot{\thepage\ of \pageref{LastPage}}

%%%%%%%%%%%%%%%%%%%%%%%%%%%%%%%%%%%%%%%%%%%%%%%
%%%%%%%%%%%%%%%%%%%%%%%%%%%%%%%%%%%%%%%%%%%%%%%

\IfFileExists{upquote.sty}{\usepackage{upquote}}{}
\begin{document}
\begin {center}

%----------------------------------------------------------------------------------------
%	YOUR NAME AND ADDRESS(ES) SECTION
%----------------------------------------------------------------------------------------

{\huge Seema Nayan Sheth}\tabularnewline
\vspace{1em}

\begin{tabularx}{\textwidth}{@{}>{\raggedright}X >{\raggedleft}X@{}}
\texttt{ssheth3}@ncsu.edu & Department of Plant and Microbial Biology \tabularnewline
Phone: (919) 515-4168 & North Carolina State University \tabularnewline
\url{http://www.seemasheth.weebly.com} & Raleigh, NC 27695 \tabularnewline
\end{tabularx}
\end{center}
\vspace{-1em}

\rule[0.5ex]{1\linewidth}{0.5pt} 

%----------------------------------------------------------------------------------------
%	PROFESSIONAL APPOINTMENTS
%----------------------------------------------------------------------------------------

\begin{flushleft}
%\vspace{0.5ex}
\section*{PROFESSIONAL APPOINTMENTS}
%\noindent\Large{\textbf{PROFESSIONAL APPOINTMENTS}} 
\vspace{-0.5ex}

{\bf Assistant Professor} \hfill {2018 - present} \newline
Department of Plant and Microbial Biology \newline
North Carolina State University
\vspace{0.5ex}

{\bf National Science Foundation Postdoctoral Fellow} \hfill {2016 - 2017} \newline
Department of Integrative Biology Biology \newline
University of California, Berkeley \newline
Supporting scientists: Dr. David Ackerly and Dr. Bruce Baldwin
\vspace{0.5ex}

{\bf Postdoctoral Research Associate} \hfill {2014 - 2015} \newline
Department of Ecology, Evolution, and Behavior \newline
University of Minnesota \newline
Advisor: Dr. Ruth Shaw
\vspace{0.5ex}

{\bf Research Scientist: Conservation, Ecology, and Evolution} \hfill {2006 - 2008} \newline
Center for Conservation and Sustainable Development \newline
Missouri Botanical Garden

%----------------------------------------------------------------------------------------
%	EDUCATION
%----------------------------------------------------------------------------------------

\vspace{1ex}
\section*{EDUCATION}
%\noindent\Large{\textbf{EDUCATION}} 
\vspace{-0.5ex}

{\bf Ph.D., Ecology} \hfill {2008 - 2014} \newline
Colorado State University \newline
Advisor: Dr. Amy Angert
\vspace{0.5ex}

{\bf M.S., Biology (Ecology, Evolution, and Systematics)} \hfill {2003 - 2006} \newline
University of Missouri - St. Louis \newline
Advisor: Dr. Bette Loiselle
\vspace{0.5ex}

{\bf B.A., Environmental Sciences and Spanish (double major)} \hfill {1998 - 2002} \newline
Washington University in St. Louis
\end{flushleft}


%----------------------------------------------------------------------------------------
%	RESEARCH INTERESTS
%----------------------------------------------------------------------------------------

\section*{RESEARCH INTERESTS}
\vspace{-0.5ex}

evolutionary and population ecology; biogeography; plant community responses to climate change

%----------------------------------------------------------------------------------------
%	PUBLICATIONS
%----------------------------------------------------------------------------------------

\vspace{1ex}
\section*{PUBLICATIONS}
%\noindent\Large{\textbf{PUBLICATIONS}} 
\vspace{-0.5ex}

Undergraduate student\textsuperscript{u}; Graduate student\textsuperscript{g}; Postdoctoral associate\textsuperscript{p} 

\begin{hangparas}{.5in}{1}
\begin{enumerate}
\reverselabel{30} %set to total # of publications

\item Sasaki, M. J. M. Barley, S. Gignoux-Wolfsohn, C. G. Hays, M. W. Kelly, A. B. Putnam, \textbf{S. N. Sheth}, A. R. Villeneuve, and B. S. Cheng. In press. More evolutionary divergence of thermal limits within marine than terrestrial species. \emph{Nature Climate Change}. Pre-print: \url{https://www.researchsquare.com/article/rs-987225/v1}.

\item Coughlin, A. O., \underline{R. Wooliver\textsuperscript{p}}, and \textbf{S. N. Sheth}. In press. Populations of western North American monkeyflowers accrue niche breadth primarily via genotypic divergence in environmental optima. \emph{Ecology and Evolution}.  

\item Wadgymar, S. M., M. L. DeMarche, E. B. Josephs, \textbf{S. N. Sheth}, and J. T. Anderson. 2022. Local Adaptation: Causal agents of selection and adaptive trait divergence. \emph{Annual Review in Ecology, Evolution, and Systematics} 53: 1. %\url{https://doi.org/10.1146/annurev-ecolsys-012722-035231}.

\item \underline{Wooliver, R.\textsuperscript{p}}, \underline{E. E. Vtipilthorpe\textsuperscript{g}}, \underline{A. M. Wiegmann\textsuperscript{u}}, and \textbf{S. N. Sheth}. 2022. A viewpoint on ecological and evolutionary study of plant thermal performance curves in a warming world. \emph{AoB Plants} 14: plac016, \url{https://doi.org/10.1093/aobpla/plac016}. 

\item \underline{Querns, A.\textsuperscript{g}}, \underline{R. Wooliver\textsuperscript{p}}, M. Vallejo-Mar\'in, and \textbf{S. N. Sheth}. 2022. The evolution of thermal performance in native and invasive populations of \emph{Mimulus guttatus}. \emph{Evolution Letters} 6: 136-148, \url{https://doi.org/10.1002/evl3.275}.

\item Lee-Yaw, J. A., McCune J. L., Pironon, S. and \textbf{S. N. Sheth}. 2022. On the predictive value of species distribution models in population biology. \emph{Ecography} 2022: e05877, \url{https://doi.org/10.1111/ecog.05877}. \textbf{(Runner-up for Ecography E4 Award)}

\item Preston, J. C, \underline{R. Wooliver\textsuperscript{p}}, H. Driscoll, A. Coughlin, and \textbf{S. N. Sheth}. 2022. Spatial variation in high temperature-regulated gene expression predicts evolution of plasticity with climate change in the scarlet monkeyflower. \emph{Molecular Ecology} 31: 1254-1268, \url{https://doi.org/10.1111/mec.16300}. 

\item Barley, J. M., B. S. Cheng, M. Sasaki, S. Gignoux-Wolfsohn, C. G. Hays, A. B. Putnam, \textbf{S. N. Sheth}, A. R. Villeneuve, and M. W. Kelly. 2021. Limited plasticity in thermally tolerant ectotherm populations: evidence for a trade-off. \emph{Proceedings of the Royal Society B} 288: 20210765, \url{https://doi.org/10.1098/rspb.2021.0765}.

\item \underline{Vtipil, E. E.\textsuperscript{u,g}} and \textbf{S. N. Sheth}. 2020. A resurrection study reveals limited evolution of phenology in response to recent climate change across the geographic range of the scarlet monkeyflower. \emph{Ecology and Evolution} 10: 14165-14177, \url{https://doi.org/10.1002/ece3.7011}.

\item \underline{Wooliver, R.\textsuperscript{p}}, S. B. Tittes, and \textbf{S. N. Sheth}. 2020. A resurrection study reveals limited evolution of thermal performance in response to recent climate change across the geographic range of the scarlet monkeyflower. \emph{Evolution} 74: 1699-1710, \url{https://doi.org/10.1111/evo.14041}.

\item \textbf{Sheth, S.N.}, N. Morueta-Holme, and A. L. Angert. 2020. Determinants of geographic range size in plants. \emph{New Phytologist} 226: 650-665, \url{https://doi.org/10.1111/nph.16406}.

\item Briscoe Runquist, R. D., A. J. Gorton, J. B. Yoder, N. J. Deacon, J. J. Grossman, S. A. Kothari, M. P. Lyons, \textbf{S. N. Sheth}, P. Tiffin, and D. A. Moeller. 2020. Context dependence of local adaptation to abiotic and biotic environments: a quantitative and qualitative synthesis. \emph{American Naturalist} 195: 412-431, \url{https://doi.org/10.1086/707322}.

\item Smithers, B. V., M. F. Oldfather, M. J. Koontz, J. Bishop, C. Bishop, J. Nachlinger, and \textbf{S. N. Sheth}. 2020. Community turnover by composition and climate affinity across scales in an alpine system. \emph{American Journal of Botany} 107: 239-249, \url{https://doi.org/10.1002/ajb2.1376}.

\item Oldfather, M. F., M. M. Kling, \textbf{S. N. Sheth}, N. C. Emery, and D. D. Ackerly. 2020. Range edges in heterogeneous landscapes: integrating geographic scale and climate complexity into range dynamics. \emph{Global Change Biology} 26: 1055-1067, \url{https://doi.org/10.1111/gcb.14897}. 

\item Lowry D. B., J. M. Sobel, A. L. Angert, T-L. Ashman, R. L. Baker, B. K. Blackman, Y. Brandvain, K. J. R. P. Byers, A. M. Cooley, J. M. Coughlan, M. R. Dudash, C. B. Fenster, K. G. Ferris, L. Fishman, J. Friedman, D. L. Grossenbacher, L. M. Holeski, C. T. Ivey, K. M. Kay, V. A. Koelling, N. J. Kooyers, C. J. Murren, C. D. Muir, T. C. Nelson, M. L. Peterson, J. R. Puzey, M. C. Rotter, J. R. Seeman, J. P. Sexton, \textbf{S. N. Sheth}, M. A. Streisfeld, A. L. Sweigart, A. D. Twyford, M. Vallejo-Marin, J. H. Willis, C. A. Wu, and Y. W. Yuan. 2019. The case for the continued use of the genus name \textit{Mimulus} for all monkeyflowers. \emph{Taxon} 68: 617-623, \url{https://doi.org/10.1002/tax.12122}. 

\item Kulbaba, M. W., \textbf{S. N. Sheth}, R. E. Pain, V. M. Eckhart, and R. G. Shaw. 2019. Additive genetic variance for lifetime fitness and the capacity for adaptation in the wild. \emph{Evolution} 73: 1746-1758, \url{https://doi.org/10.1111/evo.13830}. 

\item \textbf{Sheth, S. N.}, M. W. Kulbaba, R. E. Pain, and R. G. Shaw. 2018. Expression of additive genetic variance for fitness in a population of partridge pea in two field sites. \emph{Evolution} 72: 2537-2545, \url{https://doi.org/10.1111/evo.13614}. 

\item \textbf{Sheth, S. N.} and A. L. Angert. 2018. Demographic compensation does not rescue populations at a trailing range edge. \emph{Proceedings of the National Academy of Sciences (USA)} 115: 2413-2418, \url{https://doi.org/10.1073/pnas.1715899115}. 

\item Pain, R. E., R. G. Shaw, and \textbf{S. N. Sheth}. 2018. Costs associated with N-fixing rhizobia early in the life of partridge pea \textit{Chamaecrista fasciculata}. \emph{American Journal of Botany} 105: 796-802, \url{https://doi.org/10.1002/ajb2.1077}. 

\item Morueta-Holme, N., M. F. Oldfather, R. L. Olliff-Yang, A. P. Weitz, C. R. Levine, M. M. Kling, E. C. Riordan, C. Merow, \textbf{S. N. Sheth}, A. H. Thornhill, and D. D. Ackerly. 2018. The language of climate change: best practices in research and publication. \emph{Nature Climate Change} 8: 92-94, \url{https://doi.org/10.1038/s41558-017-0060-2}.

\item Angert, A. L., M. Bayly, \textbf{S. N. Sheth}, and J. R. Paul. 2018. Testing range-limit hypotheses using range-wide habitat suitability and occupancy for the scarlet monkeyflower (\textit{Erythranthe cardinalis}). \emph{American Naturalist} 191: E76-E89, \url{https://doi.org/10.1086/695984}.

 \item \textbf{Sheth, S. N.} and A. L. Angert. 2016. Artificial selection reveals high genetic variation in phenology at the trailing edge of a species range. \emph{American Naturalist} 187: 182-193, \url{https://doi.org/10.1086/684440}. \textbf{(American Naturalist 2016 Student Paper Award)} 
 
\item \textbf{Sheth, S. N.}, I. Jim\'enez, and A. L. Angert. 2014. Identifying the paths leading to variation in geographical range size in western North American monkeyflowers. \emph{Journal of Biogeography} 41: 2344-2356, \url{https://doi.org/10.1111/jbi.12378}. 

\item \textbf{Sheth, S. N.} and A. L. Angert. 2014. The evolution of environmental tolerance and range size: a comparison of geographically restricted and widespread \textit{Mimulus}. \emph{Evolution} 68: 2917-2931, \url{https://doi.org/10.1111/evo.12494}. 

\item \textbf{Sheth, S. N.}, L. G. Lohmann, T. Distler, and I. Jim\'enez. 2012. Understanding bias in geographic range size estimates. \emph{Global Ecology and Biogeography} 21: 732-742, \url{https://doi.org/10.1111/j.1466-8238.2011.00716.x}.
 
 \item Paul, J. R., \textbf{S. N. Sheth}, and A. L. Angert. 2011. Quantifying the impact of gene flow on phenotype-environment mismatch: a demonstration with the scarlet monkeyflower \textit{Mimulus cardinalis}. \emph{American Naturalist} 178: S62-S79, \url{https://doi.org/10.1086/661781}. 
 
\item Angert, A. L., \textbf{S. N. Sheth}, and J. R. Paul. 2011. Incorporating population-level variation in thermal performance into predictions of geographic range shifts. \emph{Integrative And Comparative Biology} 51: 733-750, \url{https://doi.org/10.1093/icb/icr048}. 

\item \textbf{Sheth, S. N.}, B. A. Loiselle, and J. G. Blake. 2009. Phylogenetic constraints on fine-scale patterns of habitat use by eight primate species in eastern Ecuador. \emph{Journal of Tropical Ecology} 25: 571-582, \url{https://doi.org/10.1017/S0266467409990216}. 

\item \textbf{Sheth, S. N.}, L. G. Lohmann, T. Consiglio, and I. Jim\'enez. 2008. Effects of detectability on estimates of geographic range size in Bignonieae. \emph{Conservation Biology} 22: 200-211, \url{https://doi.org/10.1111/j.1523-1739.2007.00858.x}. 

\item Amend, J. P., D. A. R. Meyer-Dombard, \textbf{S. N. Sheth}, N. Zolotova, and A. C. Amend. 2003. \textit{Palaeococcus helgesonii} sp. nov., a facultatively anaerobic, hyperthermophilic archaeon from a geothermal well on Vulcano Island, Italy. \emph{Archives of Microbiology} 179: 394-401, \url{https://doi.org/10.1007/s00203-003-0542-7}. 

\end{enumerate}
\end{hangparas}

%\subsection*{\textit{Manuscripts in review/revision} (available upon request)} 

%\begin{hangparas}{.5in}{1}

%Coughlin, A. M., \underline{R. Wooliver\textsuperscript{p}}, and \textbf{S. N. Sheth}. General-purpose genotypes with divergent niche optima shape population-level niche breadth in western North American monkeyflowers. In review at \emph{Evolution}. \tabularnewline

%Sasaki, M. J. M. Barley, S. Gignoux-Wolfsohn, C. G. Hays, M. W. Kelly, A. B. Putnam, \textbf{S. N. Sheth}, A. R. Villeneuve, and B. S. Cheng. Greater local adaptation to temperature in the ocean than on land. In revision for \emph{Nature Climate Change}. Pre-print: \url{https://www.researchsquare.com/article/rs-987225/v1}. \tabularnewline

%\end{hangparas}

%----------------------------------------------------------------------------------------
%	GRANTS, FELLOWSHIPS, AND AWARDS
%----------------------------------------------------------------------------------------

\vspace{2ex}
%\begin{flushleft}
\section*{GRANTS, FELLOWSHIPS, AND AWARDS}
%\noindent\Large{\textbf{GRANTS, FELLOWSHIPS, AND AWARDS}} 
%\end{flushleft}
\vspace{-0.5ex}

\renewcommand{\arraystretch}{1.2}
\begin{tabularx}{\textwidth}{@{}>{\raggedright}p{5.25in} >{\raggedleft}X@{}}

National Science Foundation, IOS, Total: \$2,236,397 (\$399,383 to NCSU) & 2023-2026 \tabularnewline
\addtolength{\leftskip}{5ex}\emph{Collaborative Research: ORCC: RUI: Integrating evolutionary and migratory potential of Chamaecrista fasciculata into forecasts of range-wide population dynamics under climate change} (\textbf{co-PI}). lead PI: Jill Anderson (U. of Georgia); co-PIs: Megan Demarche (U. of Georgia), Susana Wadgymar (Davidson College), Emily Josephs (Michigan State U.), and Jenny Cruse-Sanders (State Botanical Garden of Georgia). \tabularnewline

National Science Foundation, DEB, Total: \$1,452,695 (\$509,734 to NCSU) & 2022-2025 \tabularnewline
\addtolength{\leftskip}{5ex}\emph{Collaborative Research: BEE: Integrating evolutionary genetics and population ecology to detect contemporary adaptation to climate change across a species range} (\textbf{lead PI}). co-PIs: Chris Muir (U. of Hawai'i at M\={a}noa), Lluvia Flores-Renter\'{i}a (San Diego State U.), Jason Sexton (U. of California, Merced), and Jeff Diez (U. of Oregon). \tabularnewline

NCSU Faculty Research and Professional Development Fund, \$5,250 & 2022 - 2023 \tabularnewline
\addtolength{\leftskip}{5ex}\emph{A functional trait perspective on alpine plant community shifts in a rapidly changing climate} (PI)  & \tabularnewline

Runner-up, Ecography Award for Excellence in Ecology and Evolution (E4) & 2022 \tabularnewline

NCSU Faculty Research and Professional Development Fund, \$5,000 & 2019 - 2020 \tabularnewline
\addtolength{\leftskip}{5ex}\emph{Rapid evolution of thermal tolerance across a species' geographic range} (PI)  & \tabularnewline

American Naturalist 2016 Student Paper Award & 2017 \tabularnewline

NSF Postdoctoral Research Fellowship in Biology, \$138,000 & 2016 - 2017 \tabularnewline
\addtolength{\leftskip}{5ex}\emph{Relationships among climatic tolerance, trait evolution, and diversification in the California flora} \tabularnewline

Postdoctoral Association Career Development Award, Univ. of Minnesota, \$400 & 2015 \tabularnewline 

Finalist, University of California President's Postdoctoral Fellowship & 2014
\tabularnewline

NSF DEB Evolutionary Ecology, \$14,984 & 2012 - 2014 \tabularnewline
\addtolength{\leftskip}{5ex}\emph{Dissertation Research: Role of evolutionary potential in limiting species' distributions} (co-PI)
\vspace{0.5ex} \tabularnewline

Outreach Grant, Society for the Study of Evolution, \$800 & 2012 \tabularnewline

Global Sustainability Leadership Fellow, Colorado State University & 2012 \tabularnewline

Rosemary Grant Award, Society for the Study of Evolution, \$2,500 & 2010 \tabularnewline

% Move this chunk as needed to adjust page breaks
\end{tabularx}

\renewcommand{\arraystretch}{1.2}
\begin{tabularx}{\textwidth}{@{}>{\raggedright}p{5.25in} >{\raggedleft}X@{}}
% End of chunk

Graduate Student Research Award, Botanical Society of America, \$500 & 2010 \tabularnewline

Finalist, Environmental Protection Agency STAR Graduate Fellowship & 2009 \tabularnewline

Steinkamp Fund, Colorado Native Plant Society, \$1,000 & 2009
\tabularnewline

Women in Natural Sciences Travel Grant, Colorado State University, \$300 & 2009\tabularnewline

Awards from Department of Biology, Colorado State University, \$7,425 & 2009 - 2014\tabularnewline

Grad. Degree Program in Ecology Fellowship, Colorado State University, \$1,000 & 2008 \tabularnewline

NSF GK-12 Fellowship, Colorado State University, \$4,000 & 2008 \tabularnewline

NSF GK-12 Fellowship, Univ. of Missouri-St. Louis, \$30,000 & 2005 - 2006\tabularnewline

Awards from Department of Biology, Univ. of Missouri-St. Louis, \$3,500 & 2004 - 2005\tabularnewline

Primate Action Fund, Conservation International, \$3,000 & 2004\tabularnewline

\end{tabularx}

\vspace{0.5ex}

%----------------------------------------------------------------------------------------
%	ADD PAGE BREAK; MOVE MANUALLY IF NEEDED
%----------------------------------------------------------------------------------------
%\newpage

%----------------------------------------------------------------------------------------
%	PRESENTATIONS
%----------------------------------------------------------------------------------------
%\vspace{1.5ex}
%\begin{flushleft}
\section*{PRESENTATIONS}
Undergraduate student\textsuperscript{u}; Graduate student\textsuperscript{g}; Postdoctoral associate\textsuperscript{p} 
\vspace{-0.5ex}
\subsection*{Invited symposia}
%\nopagebreak[0]
%\end{flushleft}

\begin{tabularx}{\textwidth}{@{}>{\raggedright}p{5.25in}
>{\raggedleft}X@{}}

46th Annual Southern California Botanists Symposium (virtual) & 2020 \tabularnewline
\addtolength{\leftskip}{5ex} Living on the edge - Plants in extreme environments \tabularnewline
\addtolength{\leftskip}{5ex} ``The role of demographic and evolutionary processes in buffering populations from climate change''
\tabularnewline

International Biogeography Society Humboldt-250 Meeting, Quito, Ecuador & 2019 \tabularnewline
\addtolength{\leftskip}{5ex} Symposium: Architects of variation: How climate and physiology shape patterns of biodiversity  \tabularnewline
\addtolength{\leftskip}{5ex} ``Can plant thermal tolerance evolve under climate change? A comparison of central and edge populations'' (\underline{R. Wooliver\textsuperscript{p}}, S. Tittes, and \textbf{S.N. Sheth}; presented by R. Wooliver)
\tabularnewline

Green Life Sciences, University of Michigan, Ann Arbor, MI & 2018 \tabularnewline
\addtolength{\leftskip}{5ex} Symposium: Plant-environment interactions across scales  \tabularnewline
\addtolength{\leftskip}{5ex} ``Do demographic compensation and adaptation buffer species from changing climate?'' \tabularnewline

Society for the Study of Evolution, Portland, OR & 2017 \tabularnewline
\addtolength{\leftskip}{5ex} American Society of Naturalists Symposium: Across the Nth dimension: Quantitative and conceptual advances in the study of niche breadth  \tabularnewline
\addtolength{\leftskip}{5ex} ``Does niche breadth predict vulnerability to changing environments? From population-level traits and demography to diversification in deep time'' \tabularnewline

SACNAS, Long Beach, CA & 2016 \tabularnewline
\addtolength{\leftskip}{5ex} Scientific Symposium: (Day and) Night at the Museum: Exploring Research in Ecology and Evolution Behind the Scenes of Natural History Museums \tabularnewline
\addtolength{\leftskip}{5ex} ``Harnessing the power of herbarium specimen data for ecological and evolutionary studies'' \tabularnewline

Jornadas Argentinas de Botanica, Corrientes, Argentina & 2007 \tabularnewline
\addtolength{\leftskip}{5ex} Symposium: Conservation and Threat Assessments of Plants \tabularnewline
\addtolength{\leftskip}{5ex} ``Riesgo de extinci\'on en Bignonieae (Bignoniaceae): una perspectiva filogen\'etica'' (\textbf{Sheth, S.N.}, L.G. Lohmann, T. Consiglio, and I. Jim\'enez) \tabularnewline

% Move this chunk as needed to adjust page breaks
\end{tabularx}

\renewcommand{\arraystretch}{1.2}
\begin{tabularx}{\textwidth}{@{}>{\raggedright}p{5.25in} >{\raggedleft}X@{}}
% End of chunk

Botanical Society of America and Plant Biology Joint Congress, Chicago, IL & 2007 \tabularnewline
\addtolength{\leftskip}{5ex} \small{Colloquium: Integration of spatial and ecological data in evolutionary studies} \tabularnewline
\addtolength{\leftskip}{5ex} ``Extinction risk in Bignonieae (Bignoniaceae): a phylogenetic perspective'' (\textbf{Sheth, S.N.}, L.G. Lohmann, T. Consiglio, and I. Jim\'enez) \tabularnewline

\end{tabularx}
\vspace{-0.5ex}

%\begin{flushleft}
\subsection*{Invited seminars}
%\end{flushleft}
%\vspace{-1ex}

Genetics and Genomics Academy, North Carolina State University \hfill{2022} \newline
Department of Biology, University of Toronto, Mississauga \hfill{2022} \newline
\hspace{7mm} *** \emph{Graduate Student Invited Speaker} *** \tabularnewline
Department of Plant Biology, University of Georgia \hfill{2021} \newline
\hspace{7mm} *** \emph{Graduate Student Invited Speaker} *** \tabularnewline
Department of Ecology and Evolutionary Biology, University of Tennessee, Knoxville \hfill{2021} \newline
Department of Ecology and Evolutionary Biology, University of Colorado, Boulder \hfill{2021} \newline
Department of Ecology and Evolutionary Biology, University of California, Irvine \hfill{2021} \newline
Ecology, Evolution, and Behavior Program, Michigan State University \hfill{2021} \newline
Department of Integrative Biology, University of California, Berkeley \hfill{2020} \newline
\hspace{7mm} *** \emph{Graduate Student Invited Speaker} *** \tabularnewline
Department of Ecology and Evolutionary Biology, University of Arizona \hfill{2020} \newline
Department of Ecology and Evolutionary Biology, Tulane University \hfill {2019} \newline
Program in Ecology, Duke University \hfill {2019}\newline
Kellogg Biological Station, Michigan State University \hfill {2019} \newline
Genetics and Genomics Seminar Series, North Carolina State University \hfill {2019} \newline
Bio-Pop Seminar Series, Department of Biology, University of North Carolina \hfill {2018} \newline
Department of Plant Biology, University of Vermont \hfill {2018} \newline
EEBio Seminar Series, Department of Biology, University of Virginia \hfill {2018} \newline
Department of Forestry and Environmental Resources, North Carolina State University \hfill {2018} \newline
Department of Biological Sciences, California Polytechnic State University \hfill {2017} \newline
Department of Ecology and Evolutionary Biology, University of California, Los Angeles \hfill {2017} \newline
Environmental Systems Graduate Group, University of California, Merced \hfill {2017} \newline
School of Integrative Plant Science, Plant Biology Section, Cornell University \hfill {2017} \newline
Center for Population Biology, University of California, Davis \hfill {2017} \newline
Department of Biology, University of Utah \hfill {2017} \newline
Natural History Museum of Utah \hfill {2017} \newline
Department of Plant and Microbial Biology, North Carolina State University \hfill {2017} \newline
Department of Biology, Williams College \hfill {2016} \newline
Department of Biology, University of San Francisco \hfill {2016} \newline
Department of Biology, Grinnell College \hfill {2015} \newline
Department of Plant Biology, University of Minnesota \hfill {2015} \newline
Department of Biology, Washington University in St. Louis \hfill {2007} \newline
Department of Biology, St. Louis University \hfill {2007} \newline
\vspace{-2ex}

%\begin{samepage}

%\begin{flushleft}
\subsection*{Contributed conference presentations} %\Needspace{2in} %needspace moves section heading to previous page
%\end{flushleft}

\renewcommand{\arraystretch}{1.2}
\begin{tabularx}{\textwidth}{@{}>{\raggedright}p{5.25in} >{\raggedleft}X@{}}

\hangindent=5ex \ \underline{Goff, K. A.\textsuperscript{g}}, M. F. Oldfather, J. Nachlinger, B. Smithers, M. J. Koontz, J. Bishop, C. Bishop, and \textbf{S. N. Sheth}. ''Plant community responses to climate change over an 18-year period on alpine summits in the Sierra Nevada and Great Basin, USA.'' MtnClim, Gothic, CO (poster presented by K. A. Goff) & 2022 \tabularnewline

% Move this chunk as needed to adjust page breaks
\end{tabularx}

\renewcommand{\arraystretch}{1.2}
\begin{tabularx}{\textwidth}{@{}>{\raggedright}p{5.25in} >{\raggedleft}X@{}}
% End of chunk

\hangindent=5ex \ Lee-Yaw, J. A., McCune J. L., Pironon, S. and \textbf{S. N. Sheth}. ``How well do species distribution models predict parameters of interest in population biology?'' Ecological Society of America; Montreal, Canada (presented by J. Lee-Yaw) & 2022 \tabularnewline

\hangindent=5ex \ Preston, J. C, \underline{R. Wooliver\textsuperscript{p}}, H. Driscoll, E. Coughlin, and \textbf{S. N. Sheth}. ``Spatial variation in high temperature-regulated gene expression predicts evolution of plasticity with climate change in the scarlet monkeyflower.'' Botany (virtual; presented by J. C. Preston) & 2021 \tabularnewline

\hangindent=5ex \ Olliff Yang, R. L., \textbf{S. N. Sheth}, and D. Ackerly. ``Population differentiation in flowering time in \emph{Lasthenia gracilis}, a widespread annual forb.'' Botany (virtual; presented by R. L. Olliff Yang) & 2021 \tabularnewline

\hangindent=5ex \ Smithers, B. V., M. F. Oldfather, M. J. Koontz, J. Bishop, C. Bishop, J. Nachlinger, and \textbf{S. N. Sheth}. ``Community turnover by composition and climate affinity across scales in an alpine system.'' Ecological Society of America (virtual; presented by B. Smithers) & 2020 \tabularnewline

\hangindent=5ex \ \underline{Wooliver, R.\textsuperscript{p}}, \underline{E.E. Vtipil\textsuperscript{g}}, and \textbf{S.N. Sheth}. ``A call for unified study of plant thermal performance in a warming world.'' Botany (virtual; presented by R. Wooliver) & 2020 \tabularnewline

\hangindent=5ex \ \underline{Querns, A.\textsuperscript{g}}, \underline{R. Wooliver\textsuperscript{p}}, M. Vallejo-Mar\'in, and \textbf{S.N. Sheth}. ``The evolution of thermal performance in native and invasive populations of \emph{Mimulus guttatus}.'' Botany (virtual; poster presented by A. Querns) & 2020 \tabularnewline
\hspace{7mm} *** \emph{Winner of Best Graduate Student Poster in Ecology} *** \tabularnewline

\hangindent=5ex \ \underline{Wooliver, R.\textsuperscript{p}}, S. Tittes, and \textbf{S.N. Sheth}. ``Can plant thermal tolerance evolve under climate change? A comparison of central and edge populations.'' Southeast Population Ecology and Evolutionary Genetics, Clemson, SC (presented by R. Wooliver) & 2019 \tabularnewline

\hangindent=5ex \ \underline{Vtipil, E.E.\textsuperscript{u,g}} and \textbf{S.N. Sheth}. ``The evolution of flowering time in response to climate change in \emph{Erythranthe cardinalis}.'' Southeast Population Ecology and Evolutionary Genetics, Clemson, SC (poster presented by E. Vtipil). & 2019 \tabularnewline
\hspace{7mm} *** \textbf{Winner of Second Best Graduate Student Poster} *** \tabularnewline

\hangindent=5ex \ Kulbaba, M., \textbf{S.N. Sheth}, R.E. Pain, V.M. Eckhart, and R.G. Shaw. ``Adaptive potential and realized changes in fitness in natural populations.'' Society for the Study of Evolution, Montpellier, France (poster presented by M. Kulbaba) & 2018 \tabularnewline

\hangindent=5ex \ \textbf{Sheth, S.N.} and A.L. Angert. ``Demographic compensation does not rescue \textit{Erythranthe cardinalis} populations at the southern edge of the species range.'' Ecological Society of America, Portland, OR & 2017 \tabularnewline

\hangindent=5ex \ Kulbaba, M., R.E. Pain, V.M. Eckhart, \textbf{S.N. Sheth}, and R.G. Shaw. ``The immediate capacity for adaptation and its realization in natural plant populations.'' International Botanical Congress, Shenzhen, China & 2017 \tabularnewline

\hangindent=5ex \ \textbf{Sheth, S.N.}, M. Kulbaba, R.E. Pain, and R.G. Shaw. ``Expression of additive genetic variance for fitness in a population of partridge pea grown in two field sites.'' Society for the Study of Evolution, Portland, OR (presented by R.G. Shaw) & 2017 \tabularnewline

\hangindent=5ex \ \textbf{Sheth, S.N.}, W.A. Freyman, B.G. Baldwin, and D.D. Ackerly. ``Relationships among rates of climatic niche evolution and diversification.'' Society for the Study of Evolution, Austin, TX (poster) & 2016 \tabularnewline

% NOTE: THIS CHUNK SHOULD BE MANUALLY MOVED SO THAT PAGE BREAK IS CORRECT
\end{tabularx}
%\end{samepage}

\renewcommand{\arraystretch}{1.2} 
\begin{tabularx}{\textwidth}{@{}>{\raggedright}p{5.25in} >{\raggedleft}X@{}}
% END NOTE

% move around for correct page break
%\newpage

\hangindent=5ex \ \textbf{Sheth, S.N.} and A.L. Angert. ``Artificial selection reveals high genetic variation in phenology at the trailing edge of a species range.'' Ecological Society of America, Baltimore, MD & 2015 \tabularnewline

\hangindent=5ex \ \textbf{Sheth, S.N.} and A.L. Angert. ``Does a jack-of-all-temperatures have a large geographic range?'' Society for the Study of Evolution, Snowbird, UT & 2013 \tabularnewline

\hangindent=5ex \ \textbf{Sheth, S.N.}, I. Jim\'enez, and A.L. Angert. ``Effects of niche properties on variation in geographic range size among species of western North American monkeyflowers.'' Ecological Society of America, Portland, OR & 2012 \tabularnewline

% NOTE: THIS CHUNK SHOULD BE MANUALLY MOVED SO THAT PAGE BREAK IS CORRECT
\end{tabularx}
%\end{samepage}

\renewcommand{\arraystretch}{1.2} 
\begin{tabularx}{\textwidth}{@{}>{\raggedright}p{5.25in} >{\raggedleft}X@{}}
% END NOTE

\hangindent=5ex \ \textbf{Sheth, S.N.}, L.G. Lohmann, T. Distler, and I. Jim\'enez. ``The Wallacean shortfall: bias in estimates of geographic range size.'' Botanical Society of America, St. Louis, MO (presented by I. Jim\'enez) & 2011 \tabularnewline

\hangindent=5ex \ \textbf{Sheth, S.N.} and A.L. Angert. ``Ecological niche attributes and geographic range size in western North American monkeyflowers.'' Society for the Study of Evolution, Portland, OR & 2010 \tabularnewline

\hangindent=5ex \ \textbf{Sheth, S.N.}, L.G. Lohmann, I. Jim\'enez, and T. Consiglio. ``Riesgo de extinci\'on en Bignonieae (Bignoniaceae) estimado con datos de herbario." Congreso Latinoamericano de Bot\'anica. Santo Domingo, Dominican Republic (presented by L. G. Lohmann) & 2006 \tabularnewline 

\end{tabularx}

%----------------------------------------------------------------------------------------
%	TEACHING
%----------------------------------------------------------------------------------------
%\vspace{1.5ex}
%\begin{flushleft}
\section*{TEACHING}

\subsection*{North Carolina State University}
%\end{flushleft}
Plant Ecology, PB450/550-001, solo instructor \hfill {2019 - present} 

%\begin{flushleft}
\subsection*{Colorado State University}
%\end{flushleft}
Principles of Plant Biology Laboratory, teaching assistant \hfill {2008 - 2014} \newline
Plant Ecology, guest lecturer \hfill {2012 - 2013} \newline
Plant Ecology, teaching assistant \hfill {2011 - 2012} \newline
Biology of Organisms, guest lecturer \hfill {2009} \newline
Cache La Poudre Junior High School, NSF GK-12 fellow, LaPorte, CO \hfill {2008 - 2009}

%\begin{flushleft}
\subsection*{Missouri Botanical Garden}
%\end{flushleft}
Neotropical Plant Families, guest lecturer for University of Michigan course \hfill {2008} \newline
Conservation Biology, guest lecturer for University of Missouri - St. Louis course \hfill {2008} 

%\begin{flushleft}
\subsection*{University of Missouri - St. Louis}
%\end{flushleft}
McCluer High School, NSF GK-12 fellow, Florissant, MO \hfill {2005 - 2006} \newline
Organisms and the Environment Laboratory, guest lecturer \hfill {2005} 

%\begin{flushleft}
\subsection*{AmeriCorps Partnership for Youth}
%\end{flushleft}
Woodward Elementary School, tutor and teaching assistant, St. Louis, MO \hfill {2002 - 2003} 

%----------------------------------------------------------------------------------------
%	MENTORING
%----------------------------------------------------------------------------------------
\section*{MENTORING}

%\renewcommand{\arraystretch}{1.2}
%\begin{tabularx}{\textwidth}{@{}>{\raggedright}p{5.25in} >{\raggedleft}X@{}}

\subsection*{North Carolina State University}
\subsubsection*{\emph{Postdoctoral associates}}
Dr. Rachel Wooliver \hfill {2018 - 2020}
\subsubsection*{\emph{Graduate students}}
Kaleb Goff (Ph.D.) \hfill {2021 - present} \newline
Emma Vtipilthorpe (Ph.D.) \hfill {2020 - 2021} \newline
Aleah Querns (M.S.) \hfill {2018 - 2020} \newline
Emma Vtipil (M.R.) \hfill {2019 - 2020}

\subsubsection*{\emph{Undergraduate researchers}}
Devin Adas \hfill {2021} \newline
Brooke Caldwell \hfill {2019 - 2020} \newline
Mataeus Funderburk \hfill {2022 - present} \newline
Natalie Gold \hfill {2019} \newline
Mariah Kidd \hfill {2019 - 2020} \newline
Sophie Meng \hfill {2021 - present} \newline 
Daisy Ryan \hfill {2020 - 2021} \newline 
Emily Powell \hfill {2021} \newline 
Jessie Torres \hfill {2018 - 2019} \newline
Emma Vtipil (Honor's thesis student; recipient of NCSU Chilton Research Award) \hfill {2018 - 2019} \newline
Mia Wiegmann \hfill {2018 - 2021} \newline 
Emma Wilson \hfill {2021 - present} \newline 
Collin Yurish \hfill {2018 - 2019} 

\subsubsection*{\emph{Graduate student committees}}
Erin Eichenberger, NCSU \hfill {2020 - present} \newline 
Samuel Flake, NCSU \hfill {2018 - 2021} \newline
Kira Lindelof, NCSU \hfill {2020 - present} \newline
Ryan O'Connell, Duke University \hfill {2020 - present} \newline
Simon Pinilla-Gallego, NCSU \hfill {2018 - 2021} \newline
Anita Simha, Duke University \hfill {2020 - present} \newline
Greg Wilson, NCSU \hfill {2018 - 2019} 

\subsection*{University of California, Berkeley}
\hangindent=5ex Mentored 7th graders at King Middle School as part of Be a Scientist outreach program \hfill {2017} 

\subsection*{University of Minnesota}
Sam Weaver (NSF REU student from St. Olaf College)		 \hfill {2015} 

\subsection*{Colorado State University}
Amber Weimer (Honor's thesis undergraduate student)		 \hfill {2013 - 2014} \newline
Trained, mentored, and supervised 17 undergraduates \& recent college graduates		 \hfill {2009 - 2014} \newline
Biological Summer Undergraduate Research Experience program committee member \hfill {2012} 

%----------------------------------------------------------------------------------------
%	SERVICE
%----------------------------------------------------------------------------------------
\vspace{1.5ex}
%\begin{flushleft}
\section*{SERVICE}
\vspace{-0.5ex}
%\end{flushleft}
%\begin{flushleft}
\subsection*{University service}
Plant Biology Graduate admissions committee, NCSU \hfill{2018 - present} \newline
College of Agricultural and Life Sciences greenhouse committee, NCSU \hfill{2019 - present}\newline
Ecology faculty search committee, Dept. of Plant and Microbial Biology, NCSU \hfill{2019 - 2020 }\newline
Biology faculty search committee, Colorado State University \hfill{2012 - 2013} 

\subsection*{Professional service}
\renewcommand{\arraystretch}{1.2}
\begin{tabularx}{\textwidth}{@{}>{\raggedright}p{5.25in} >{\raggedleft}X@{}}
\hangindent=5ex 
Chair, Workshop and Regional Society Committee, American Society of Naturalists & 2021 - present \tabularnewline

% NOTE: THIS CHUNK SHOULD BE MANUALLY MOVED SO THAT PAGE BREAK IS CORRECT
\end{tabularx}
%\end{samepage}

\renewcommand{\arraystretch}{1.2} 
\begin{tabularx}{\textwidth}{@{}>{\raggedright}p{5.25in} >{\raggedleft}X@{}}
% END NOTE

Manuscript reviewer: \emph{American Journal of Botany, American Naturalist, Annals of Botany, AoB PLANTS, Biology Letters, BMC Evolutionary Biology, Current Biology, Ecography, Ecological Applications, Ecology, Ecology Letters, Evolution, Evolutionary Ecology, Functional Ecology, Journal of Animal Ecology, Journal of Ecology, Molecular Ecology, New Phytologist, Philosophical Transactions of the Royal Society B, PNAS, Proceedings of the Royal Society B, Science Advances, Trends in Ecology and Evolution} & 2009 - present \tabularnewline

Panelist: NSF Division of Environmental Biology & 2022 \tabularnewline    
Proposal reviewer: NSF Division of Environmental Biology & 2021, 2022 \tabularnewline    
Book proposal reviewer: Oxford University Press & 2018 \tabularnewline                               
Proposal reviewer: NSF Division of Environmental Biology, Population and Community Ecology & 2014 
\end{tabularx}

\subsection*{Other service}
\renewcommand{\arraystretch}{1.2}
\begin{tabularx}{\textwidth}{@{}>{\raggedright}p{5.25in} >{\raggedleft}X@{}}
\hangindent=5ex Volunteer: Global Observation Research Initiative in Alpine Environments (GLORIA) Great Basin, California and Nevada & 2017 - present \tabularnewline
\hangindent=5ex Invited Panelist, Community Dialogue on Asian Americans and Pacific Islanders in Ecology and Evolutionary Biology (virtual), Northeastern University Marine Science Center & 2022 \tabularnewline 
\hangindent=5ex Invited Panelist, Forestry and Environmental Resources Leadership Board's Research Panel and Mixer, NCSU & 2022 \tabularnewline 
\hangindent=5ex Invited Panelist, PhD Career Panel (virtual), Graduate Degree Program in Ecology, Colorado State University & 2021 \tabularnewline
\hangindent=5ex Exhibitor, Darwin Day, North Carolina Museum of Natural Sciences & 2019, 2021 \tabularnewline
\hangindent=5ex Primary Atlaser (Cary SE Priority Block) and Volunteer: North Carolina Bird Atlas & 2021 - present \tabularnewline
\hangindent=5ex Judge, Southeast Population Ecology and Evolutionary Genetics & 2019 \tabularnewline
\hangindent=5ex Advisory board member: Science Ambassador Scholarship (for undergraduate women in science, technology, engineering, or math), Cards Against Humanity & 2017 - 2018 \tabularnewline

\hangindent=5ex Collaborated with National Park Service to recruit Navajo Nation students to assist with fieldwork in Canyon de Chelly National Monument, AZ & 2009 \tabularnewline			
Volunteer, Putnam Elementary Science Carnival, Fort Collins, CO & 2009 \tabularnewline                                
Volunteer, Unidad T\'ecnica (environmental NGO), Managua, Nicaragua &2001 \\*[0.7em]
%\end{flushleft}
\end{tabularx}
 
%----------------------------------------------------------------------------------------
%	ADD PAGE BREAK; MOVE MANUALLY IF NEEDED
%----------------------------------------------------------------------------------------
%\newpage

%----------------------------------------------------------------------------------------
%	WORKING GROUPS AND WORKSHOPS
%----------------------------------------------------------------------------------------
%\vspace{1.5ex}
%\begin{flushleft}
\vspace{-2ex}
\begin{samepage}
\section*{WORKING GROUPS AND WORKSHOPS} %\Needspace{2in} %needspace moves section heading to previous page
\vspace{-0.5ex}

\renewcommand{\arraystretch}{1.2} 
\begin{tabularx}{\textwidth}{@{}>{\raggedright}p{5.25in} >{\raggedleft}X@{}}

\hangindent=5ex Working group on evolutionary processes in Long-Term Ecological Research sites (Evo-LTER), Sevilleta National Wildlife Refuge LTER, La Joya, NM & 2022 \tabularnewline 

\hangindent=5ex Mentoring Makes a Difference Workshop Series, North Carolina State University (virtual) & 2022 \tabularnewline

\hangindent=5ex HHMI Inclusive Excellence Faculty Workshop, North Carolina State University, Raleigh, NC & 2021 \tabularnewline

\hangindent=5ex Agricultural Leadership Learning Institute for Faculty, North Carolina State University, Raleigh, NC & 2019 \tabularnewline 

\hangindent=5ex Evolution in Changing Seas Synthesis Workshop, Shoals Marine Laboratory, Appledore, ME & 2019 \tabularnewline  

% NOTE: THIS CHUNK SHOULD BE MANUALLY MOVED SO THAT PAGE BREAK IS CORRECT
\end{tabularx}
\end{samepage}

\renewcommand{\arraystretch}{1.2} 
\begin{tabularx}{\textwidth}{@{}>{\raggedright}p{5.25in} >{\raggedleft}X@{}}
% END NOTE

\hangindent=5ex Software Carpentry Workshop, Berkeley Institute of Data Sciences, Berkeley, CA & 2016 \tabularnewline

\hangindent=5ex Early Career Centennial Mentoring Program, Ecological Society of America, Baltimore, MD & 2015 \tabularnewline

\hangindent=5ex Quantitative Genetics and Mixed Models in Quantitative Genetics, Summer Institute in Statistical Genetics, University of Washington, Seattle, WA & 2013 \tabularnewline    

\hangindent=5ex Intro to Python for ArcGIS Workshop, Colorado State University, Fort Collins, CO & 2013 \tabularnewline

\hangindent=5ex Science Communication Workshop, Colorado State University, Fort Collins, CO	& 2012 \tabularnewline

\hangindent=5ex Living on the edge: integrating science into the management of range-margin populations, University of Wyoming, Laramie, WY & 2010 \tabularnewline			

\hangindent=5ex Applied Phylogenetics Workshop, Bodega Bay Marine Laboratory, CA & 2010 \tabularnewline

\hangindent=5ex Working with ArcGIS Spatial Analyst (short course), Environmental Systems Research Institute, St. Charles, MO & 2007 \tabularnewline

\hangindent=5ex Distance Sampling Workshop, University of Missouri - St. Louis, St. Louis, MO & 2006 \tabularnewline

\hangindent=5ex Assessing extinction risk in Bignonieae (summer internship), Center for Conservation and Sustainable Development, Missouri Botanical Garden, St. Louis, MO & 2005 \tabularnewline

\hangindent=5ex Geographic Information Systems (course), University of Missouri - St. Louis & 2004 \tabularnewline

\hangindent=5ex Tropical Biology: An Ecological Approach (course), Organization for Tropical Studies, Costa Rica & 2004 \tabularnewline

\hangindent=5ex Historical Biogeography (short course), University of Missouri - St. Louis & 2004 \tabularnewline

\end{tabularx}

%\end{flushleft}

%----------------------------------------------------------------------------------------
%	PROFESSIONAL AFFILIATIONS AND SOCIETIES
%----------------------------------------------------------------------------------------
%\vspace{1.5ex}
%\begin{flushleft}
%\vspace{-2ex}
\section*{PROFESSIONAL AFFILIATIONS AND SOCIETIES}
%\vspace{-0.5ex}
%\end{flushleft}

\renewcommand{\arraystretch}{1.2}
\begin{tabularx}{\textwidth}{@{}>{\raggedright}p{5.25in} >{\raggedleft}X@{}}

\hangindent=5ex Science Director: GLORIA (Global Observation Research Initiative in Alpine Environments) Great Basin \tabularnewline
Faculty Affiliate: Southeast Climate Science Center \tabularnewline
American Society of Naturalists \tabularnewline
Botanical Society of America \tabularnewline
Ecological Society of America \tabularnewline
Society for the Study of Evolution \tabularnewline
\end{tabularx}
\vfill
\begin{center}
\textit{last updated \today}
\end{center}
\end{document}
